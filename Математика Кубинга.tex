\documentclass{book}
\usepackage[utf8]{inputenc}
\usepackage[english,russian]{babel}
\usepackage{amsthm}
\usepackage{amssymb}
\usepackage{amsmath}
\usepackage{mathtools} % Для \mathclap
\usepackage{geometry} % Для настройки полей
\usepackage{fancyhdr} % Для работы с колонтитулами
\usepackage{titlesec} % Для настройки заголовков
\usepackage{setspace} % Для межстрочного интервала
\usepackage{array}    % Для форматирования столбцов
\usepackage{longtable} % Для многостраничных таблиц
\usepackage{etoolbox} % Для работы с макросами
\usepackage{hyperref}

% Настройка полей
\geometry{a4paper,left=4cm,right=4cm,top=3cm,bottom=3cm}

% Работа с терминами и глоссарием
\makeatletter
\newcommand{\glossarylist}{}
\newcounter{termcounter}

\newcommand{\newterm}[2]{%
  \refstepcounter{termcounter}%
  \textbf{#1}%
  \textit{ — #2}%
  \textnormal{ \textit{[терм.\thetermcounter]}}%
  \label{term:\thetermcounter}%
  \protected@edef\glstemp{%
    \noexpand\gappto\noexpand\glossarylist{%
      \unexpanded{\bfseries #1} & 
      \unexpanded{#2} & 
      (\thetermcounter, стр. \noexpand\pageref{term:\thetermcounter})\noexpand\\
      \noexpand\hline
    }%
  }%
  \glstemp
}

\newcommand{\printglossary}{%
  \begin{longtable}{
    >{\raggedright\arraybackslash}p{0.25\linewidth}
    >{\raggedright\arraybackslash}p{0.6\linewidth}
    >{\raggedleft\arraybackslash}p{0.15\linewidth}
  }
    \hline
    \textbf{Термин} & \textbf{Определение} & \textbf{Стр. (нажать, чтобы перейти)} \\
    \hline
    \endhead
    \glossarylist
  \end{longtable}%
}
\makeatother

% Работа с пометками
\newcounter{remarkcounter}
\newcommand{\REMARK}[1]{%
    \par\noindent%
    \stepcounter{remarkcounter}%
    \textbf{ПОМЕТКА \theremarkcounter:}%
    \par\noindent%
    \em%
    \let\oldtextbf\textbf%
    \let\oldem\em%
    \renewcommand{\textbf}[1]{\oldem\oldtextbf{##1}}%
    #1%
    \let\textbf\oldtextbf%
    \let\em\oldem%
    \normalfont%
}

% Настройка колонтитулов
\pagestyle{fancy}
\fancyhf{}
\renewcommand{\chaptermark}[1]{\markboth{\chaptername\ \thechapter. #1}{}}
\renewcommand{\sectionmark}[1]{\markright{\thesection. #1}}
\fancyhead[LE]{\leftmark}
\fancyhead[RO]{\rightmark}
\fancyhead[RE,LO]{\thepage}

% Заголовок и оглавление
\title{МАТЕМАТИКА КУБИНГА}
\author{Andy Kybik}
\date{\today}
\setcounter{tocdepth}{5}

\begin{document}

\maketitle
\tableofcontents

\chapter{Формат документа}

В документе использованы ссылки, формат которых - \textit{[терм. N]} и №страницы. На данный момент (16 апреля 2025) это односторонние ссылки, и используются они только для определений. Формат ссылки - с одной стороны неинтерактивная \textit{[term N]}, с другой - интерактивная \textit{[term n] , стр.№N} в оглавлении, ведующая к определению термина в основной части документа. На интерактивные ссылки можно нажимать и автоматически отматывать документ на нужное место. Впрочем, даже если автоматика по каким-то причинам не работате, в глосссарии указана нужная страница для каждого террмина. Ссылки для терминов в основной части документа неинтерактивны, так как в любом случае ведут в глоссарий.

Ссылки в оглавлении также интерактивны.

\chapter{Предисловие}

Многие термины в математике, а особенно базовые, могут не иметь определений (такие как точка, прямая, плоскость, гиперплоскость...) или иметь определения, порождающие циклы. 

Один из таких логических циклов я встретил, будучи на 2-ом курсе МИЭТ, когда лектор по Численным Методам упомянул, что прямую можно определить как траекторию точки, движущейся из бесконечности в бесконечность согласно вектору. Меня это сильно смутило, и после лекции мы завели спор о том, что в математике должно определяться первым - точка или вектор - ведь вектор - это направленный отрезок, отрезок - часть прямой, а прямая - состоит из точек и может быть определена в свою очередь так, как это сделал он. Этот спор стал настолько фундаментальным, удивительным и значимым для меня, что я помню его и по сей день, хотя мы так и не дошли до какой-либо четкой формы истины.

В данном документе определены термины кубинга. Определены они максимально универсально, так, чтобы это было верно для самых разных случаев и не зависело от деталей. Однако многие термины могут порождать примеры таких логических циклов в своих определениях, ссылаясь друг на друга. С одной стороны, я постараюсь изложить их в порядке, требующем минимального числа опережающих ссылок. С другой же - я не страшусь их и время от времени все же буду (причем намеренно) упоминать о том, что такой-то термин может быть значимым в контексте употребления с термином, определенным далее.

В сумме данный документ - набросок для книги, труда по Математике Кубинга и ставит одной из целей его распространение.

\chapter{Определения}

\section{Стикер}

\newterm{Стикер}{максимально аморфное, не имеющее четкого определения понятие, которое, впрочем, имеет значение и применяется, является критерием совпадения одних и других частей пазлов в кубинге, будь то элементы или, например, блоки.}

Стикеры могут иметь 
\begin{enumerate}
    \item любой цвет (как на кубике 3×3×3, например, белый или зеленый),
    \item любую расцветку (как, например, в 3×3×3 Rainbow или на химерах со значащей ориентацией центров),
    \item любую форму (как, например, у Аксиса или Бампедов),
    \item любую мерность (как это различается у 3D-пазлов, где стикер, хоть и может быть неровным, как у Аксиса, но все же представляет двумерные многообразия, и у $(N\geq3)D$-пазлов, где стикеры обычно трехмерны, хотя технически могут быть математически преобразованы в точки с разными индексами или другими отличающими характеристиками).
\end{enumerate}

При собранном состоянии каждый стикер находится на своем законном положении. Технически в данной системе обозначений положение также обозначается через понятие стикера, а нахождение стикера на своем месте достигается через равенство

\begin{equation}
    \text{Stic}_A = \text{Stic}_B
\end{equation}

стикера А и его места В, обозначенного также как стикер.

Тем же "трюком" мы будем пользоваться, чтобы обозначить расположение элемента на своем месте.

\section{Элемент}

\newterm{Элемент}{множество стикеров, которые `путешествуют' всегда только вместе.} Это значит, что

\begin{enumerate}
    \item если группа стикеров разъединяется хотя бы одной операцией (например, для большинства головоломок - поворотом), то они принадлежат разным элементам. Такое определение (а точнее, часть определния) неконструктивное, так как лишь дает понять, является ли данное множество стикеров элементом. Имеет смысл озадачиться конструктивным определением, а скорее даже - функцией, вычисляющей конкретно, к каким элементам относятся данные стикеры, и какие из них на какую часть составляют.
    \item если группа стикеров `неполна', то есть, к ним можно добавить еще, как минимум, 1 стикер, который также не отъединяется от них ни одной операцие, то она не составляет полный элемент и в обязательном порядке необходимо добавить этот (эти) стикер(ы).
\end{enumerate}

\begin{equation}
    \begin{gathered}
    \text{Element}_A = \bigcup_{i\in\mathbb{N}} \text{Stic}_i, \quad \nexists \, \text{Op}_k(\text{Element}_A) \to \text{Element}_B = \bigcup_{j\in\mathbb{N}} \text{Stic}_j : \\
    \left[
    \begin{gathered}
    \exists \, \text{Stic}_j \notin \text{Element}_A \\
    \exists \, \text{Stic}_i \notin \text{Element}_B
    \end{gathered}
    \right.
    \end{gathered}
\end{equation}

\section{Слой}

\newterm{Слой}{все элементы, смещаемые одной операцией. При этом, если есть операция, смещающая подножество только из данных элементов, но не все, значит прежнее множество не образовывало слой, и оно может быть разбито, как минимум, на две части, как минимум, одна из которых точно образует меньший слой.}

\begin{equation}
    \text{Layer}_A = \bigcup_{i\in\mathbb{N}} \text{Element}_i, \quad \exists \, \text{Op}: \forall \text{Element}_i \to \text{Element}_j \in \text{Layer}_A
\end{equation}

\section{Ось}

\newterm{Ось}{все слои, параллельные данному.} \\ 

\REMARK{На самом деле определение ниже указывает на непересечение, а не параллельность, что можно проследить на примере мегаминкса - ведь у него есть слои, не пересекающиеся с данным, но и не параллельные ему. Проблема была бы решена, если была бы введено понятие Полноты пазла в элементах и слоях (название можно уточнять) - наличие всех элементов, возникающих при комбинаторном пересечении всех слоев всех осей. В этом случае конкретно касательно мегаминкса, он превартился бы в звездчатый додекаэдр и имел порядка 200 элементов. Тогда определение ниже было бы точным для этого случая.}

Формой `полного' мегаминкса выступает звездчатый додекаэдр, но лишь в случае, если каждый элемент образован геометрически (а точнее, стереометрически) в виде пересечения слоев одинаковой толщины. Иными словами, если не шейпмод `полного' мегаминкса. Коль угодно достроить звездчатый додекаэдр до платонова тела, можно изменить форму элементов, стоящих на вогнутостях, и получить икосаэдр. \\

\REMARK{Ввести понятие геометрически правильного пазла и его шейпмода, зацепившись за изложенную выше идею. Такое определение было бы более точным, чем историческое название многих шейпмодов по факту, и даже могло бы конфликтовать с некоторыми из них в каких-то особенных случаях.}

\begin{equation}
    \begin{gathered}
    \text{Axe} = \bigcup_{i\in\mathbb{N}} \text{Layer}_i : \forall \text{Layer}_i \, \nexists \, \text{Element}_k \in \text{Layer}_j : \text{Element}_k \in \text{Layer}_A \\
    \Updownarrow \\
    \text{Layer}_i \parallel \text{Layer}_j
    \end{gathered}
\end{equation} \\

\REMARK{Ось не обязана `захватывать' абсолютно все детали пазла. Например, могут быть пазлы, имеющие или особые элементы в собранный момент изначально, или особые состояния, когда некоторые элементы остаются неподвижными при воздействии на пазл любых операций. Возможно, имеет смысл разделить пазлы на два класса:

\begin{enumerate}
    \item для которых каждая ось `содержит' весь пазл
    \item для которых есть некоторые (хотя бы 1) оси, не `захватывающие' некоторые элементы
\end{enumerate}

Впоследствии каждый тип может иметь свои уникальные свойства, хотя главная задача Математики Кубинга - обобщить все пазлы как единую Модель и подвести их все под общий метод-знаменатель.}

\section{Операция}

\newterm{Операция}{функция над элементом, перемещающая его на место другого элемента при том, что все стикеры обоих элементов взаимно обмениваются.} Это значит, что 

\begin{enumerate}
    \item ни один стикер одного элемента не может `затеряться' в другом, третьем
    \item нет ни одного стикера, который, наоборот бы, приходил с какого-то третьего элемента
    \item то есть суммарно это можно назвать комбинаторно-вариативной биекцией, сохраняющей структуру множества стикеров на одном и другом элементе, но дозволяющую их перестановку в рамках одного элемента
    \item \REMARK{При этом имеет смысл ввести понятие сохраенения ориентации на множестве стикеров каждого элемента при проведении опрации. Это важно, так как есть опасность, например, на 4D-кубике стикеры возымеют 4 стикера и невозможность быть переставленными определенным образом (а именно крест на крест). При росте D это явление будет множиться, причем нелинейно. Имеет смысл посчитать число таких невозможных комбинаций}
\end{enumerate}

\begin{equation}
    \text{Op}_A(\text{Layer}_A) = f(\text{Layer}_A) : \forall f(\text{Element}_i \in \text{Layer}_A) \to \text{Element}_j \in \text{Layer}_A
\end{equation}

Также введем отдельно на всякий случай нулевую операцию:
\newterm{Нулевая операция}{операция, сохраняющая элемент/блок/слой (множество элементов) в покое.}

\begin{equation}
    \text{Op}_0(X) = f(X) : X \to X
\end{equation}

\section{Замес}

\newterm{Замес (или скрамбл)}{структура из операций, которые переводят некоторые элементы пазла в некоторые другие, принадлежащие ему, и при этом обязательно меняющая состояние пазла (стало быть, число операций ненулевое).} \\

\REMARK{При этом не указано, какого рода данное множество операций, какого оно характера, природы, и каковы причины выбора именно такого его. Например, это может быть
\begin{enumerate}
    \item Путь из собранного состояния в замес, то есть цепочка последовательных перестановок, ведущих к текущему состоянию, приводящих к нему. В таком случае замес рассматривается как процесс
    \item Набор циклов или минперов, указывающих минимальную или менее выгодную разницу между собранным состоянием и данным замесом. В таком случае замес рассматривается как результат, что сильно экономит память и эффективность алгоритма вычисления данного замеса.
\end{enumerate}
}

\begin{equation}
    \text{Scrumble}\left(X = \bigcup_{i\in\mathbb{N}} \text{Element}_i\right) =
    \left\{
    \begin{gathered}
    \bigcup_{i\in\mathbb{N}} \text{Op}_i(\text{Element}_i) \to \text{Element}_j \in X \\
    \text{Scrumble}(X) \neq X
    \end{gathered}
    \right.
\end{equation}

\section{Псевдо-замес}

\newterm{Псевдо-замес (или псевдо-скрамбл)}{}

\section{Блок}

\newterm{Блок}{} 

\begin{equation}
    \text{Block} = \bigcup_{i\in\mathbb{N}} \text{Element}_i : \nexists \, \text{Scrumble}(\text{Block})
\end{equation}

\section{Клок}

\newterm{Клок}{} 

\section{Ус}

\newterm{Ус}{} 

\section{Ограничитель}

\newterm{Ограничитель}{} 

\section{Пазл/паззл}

\REMARK{Допускается написание как через одну `з', так и через две}

\newterm{Пазл/паззл}{}

\begin{equation}
    \text{Puzzle} = \bigcup_{i\in\mathbb{N}} \text{Element}_i
\end{equation}

\section{Бандажинг}

\newterm{Бандажинг}{}

\section{Разандажинг}

\newterm{Разандажинг}{}

\section{Бандаж}

\newterm{Бандаж}{}

\section{Геометрическая бандажная форма}

\newterm{Геометрическая бандажная форма}{учесть псевдо-блоки!}

\section{Цветная бандажная форма}

\newterm{Цветная бандажная форма}{учесть псевдо-блоки!}

\section{Свободные элементы}

\newterm{Свободные элементы}{}

\section{Псевдоблоки}

\section{Блокбилдинг}

\newterm{Блокбилдинг}{} 

\newterm{Псевдоблоки}{свободные элементы, временно называемые и алгоритмически воспринимаемые как блоки}

\section{Стикермод}

\newterm{Стикермод}{}

\section{Динамичный бандаж}

\newterm{Динамичный бандаж}{}

\begin{equation}
    \begin{gathered}
    \text{Bandage} = \text{Puzzle} : \exists \, \text{Scrumble}_X(\text{Puzzle}) : \\
    \bigcup_{i\in\mathbb{N}} \text{Op}_i(\text{Element}_i \in \text{Scrumble}_X(\text{Puzzle})) \neq \bigcup_{i\in\mathbb{N}} \text{Op}_i(\text{Element}_i \in \text{Puzzle})
    \end{gathered}
\end{equation}

\section{Минпер}

\newterm{Минпер}{}

\section{Паритет}

\newterm{Паритет}{}

\section{Конжугат}

\newterm{Конжугат}{}

\section{Сетап}

\newterm{Сетап}{}

\section{Коммутатор (комм)}

\newterm{Коммутатор (комм)}{}

\section{Цикл}

\newterm{Цикл}{}

\section{Конжекомм}

\newterm{Конжекомм}{}

\section{Метод}

\newterm{Метод}{}

\section{Псевдо-антипод}

\newterm{Псевдо-антипод}{}

\section{Алгоритм Бога (Алг Бога/АБ)}

\newterm{Алгоритм Бога (Алг Бога/АБ)}{}

\section{Число Бога (ЧБ)}

\newterm{Число Бога (ЧБ)}{}

\section{Перечень Бога (ПБ)}

\newterm{Перечень Бога (ПБ)}{}

\section{Метод Бога (МБ)}

\newterm{Метод Бога (МБ)}{}

\section{Антипод}

\newterm{Антипод}{}

\section{Алг Дьявола (АД)}

\newterm{Алг Дьявола (АД)}{}

\chapter{Теоремы}

\chapter{Глоссарий}
\printglossary

\end{document}
